%!TEX root = ../main.tex

\chapter{Digitalizzazione di audio analogico}
\label{chp:digitalizzazione}

% Random citation \cite{russo_enhancing_nodate}. \\
% Random footnote.\footnote{\url{https://lucamartinelli.eu.org}}

% TODO specifica quando, non a caso?
Negli ultimi anni in tutto il mondo si è sentita la necessità di preservare dall'oblio il proprio patrimonio culturale.

Al contrario della conservazione passiva, che consiste nella sola salvaguardia dei documenti nella loro dimensione fisica; l'unica soluzione per preservare a lungo termine il materiale analogico è la digitalizzazione, ovvero una conservazione di tipo attivo.
Tra il materiale da digitalizzare, quello audio-visivo è sicuramente uno dei più "ingannevoli" per via della necessità di preservare sia il suo stato corrente, seppur rovinato dal tempo o dall'utilizzo, compreso di tutte le informazioni accessorie che di disporre di una copia "di accesso" in maniera tale da essere riprodotta semplicemente a fini divulgativi.

Molte organizzazioni quali archivi e librerie non hanno ancora messo in pratica misure atte a preservare a tempo indeterminato il loro patrimonio culturale e ciò si verifica non solo in nazioni in via di sviluppo \cite{rakemaneChallengesManagingPreserving2021}, ma anche in Italia \cite{raimoDigitalizationCulturalIndustry2022} presentando criticità non dissimili. % mancanza di fondi e ancora poca importanza riposta, anche loro hanno trovato l'istituzione ed implementazione di politiche di conservazione come soluzione

La digitalizzazione, inoltre, permette ai musei di creare esperienze online che, negli anni e soprattutto con l'avvento del COVID-19, si sono dovuti adeguare introducendo la tecnologia digitale dapprima nelle loro strutture per migliorare l'esperienza dei visitatori e poi anche in rete creando delle mostre virtuali generando anche maggiori visite ed incassi oltre che riducendo i costi \cite{raimoDigitalizationCulturalIndustry2022}. % TODO perche?
Questo sforzo è anche in linea con le raccomandazioni dell'UNESCO \cite{unescoRecommendationConcerningProtection} in riferimento all'importanza della tecnologia in ambito educativo e culturale.

% TODO mostrare meglio la differenziazione tra audio e generico
\section{Filologia e Fedeltà}
La maniera più basilare di procedere alla digitalizzazione è registrare soltanto l'informazione primaria ovvero, nel caso di un prodotto sonoro, il segnale audio.
Per digitalizzare, invece, il prodotto in maniera filologicamente corretta è necessario memorizzare anche le informazioni ausiliarie come scritte sul contenitore, rumori presenti sul sistema di registrazione, alterazioni fisiche del supporto ed altri metadati oltre alla storia del tramandamento del documento (archiviazione, duplicazione, ecc.) \cite{prettoComputingMethodologiesSupporting2018}.

Per ottenere il massimo livello di aderenza alla realtà: la fedeltà, si deve assicurare, oltre che una riproduzione audio fedele, anche una simulazione dell'esperienza di interazione col dispositivo di riproduzione ed una riproduzione dei metadati e delle informazioni contestuali \cite{fantozziTapeMusicArchives2017}.

Ad onor del vero in certi casi può essere utile operare delle modifiche alla traccia digitalizzata nel caso in cui ci siano degli errori o delle corruzioni nel prodotto iniziale o in maniera tale da creare una versione utilizzabile in maniera più semplice con le tecnologie moderne, a discapito della fedeltà; una cosiddetta copia di accesso.

\section{Digitalizzazione manuale e automatizzazione}
Il trasferimento da analogico a digitale dipende ancora dall'esperienza dell'operatore (dalle sue valutazioni e scelte se intervenire o meno), ciò comporta l'introduzione di errori indesiderati causati dalla perdita di attenzione umana in seguito a numerose ore di lavoro con riflessi negativi sul valore dei documenti creati e sull'affidabilità dell'intera collezione.
Per questo motivo è importante l'automatizzazione delle attività ripetitive.