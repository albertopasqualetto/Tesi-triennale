%!TEX root = ../main.tex

\chapter{Conclusioni} \label{chp:conclusioni}
In conclusione si è visto perché e come conservare opere musicali seguendo la traccia del software del \ac{CSC}.
I test di conformità sono stati scritti seguendo il documento che li descrive; talvolta si è percepita la pesantezza dell'articolata struttura di \ac{MPAI} per via dei molteplici documenti, anche ridondanti, che vengono prodotti, ma ciò probabilmente è dovuto alle dimensioni, le estensioni e la voglia di crescere dell'organizzazione. Tale "pesantezza" non è, però, avvertita nello sviluppo software, che invece è gestito dai vari sottogruppi.

Non sono stati riscontrati problemi irrisolti durante lo sviluppo del codice.


\section{Cosa manca}  % gli altri AIMs e usare altri dataset per migliorare le soglie
Per completare i lavori mancano i test di conformità per gli altri 3 \acp{AIM}, che non sono stati codificati per limitazioni di tempo.

Inoltre sarebbe opportuno ritoccare le soglie del controllo di uguaglianza per audio e video nei test del Packager (sottosezioni \ref{ssec:packager-audio} e \ref{ssec:packager-video}) potendo disporre di più pacchetti di datasets.


\section{Obiettivi futuri}  % audio enhancement in input o basato su irregolarità, risvolti sulla fedeltà
Come anticipato nella relativa sottosezione \ref{ssec:packager-video}, si può studiare un metodo più preciso nel confrontare i flussi audio rispetto a quello presentato.

Inoltre, ai fini di migliorare il prodotto finale si potrebbe:
\begin{itemize}
    \item \begin{onehalfspace}
        Aggiungere uno stadio di miglioramento dell'audio rimuovendo le alterazioni eventualmente incontrate nella traccia audio (click, rumore, \dots) \cite{godsillDigitalAudioRestoration1998}.
        Una proposta simile era già stata ipotizzata in \ac{CAE}, come si può vedere dalla presenza di un \ac{AIM} chiamato "Audio Enhancer" in \cite{mpaiMPAICAEUseCases2021} nel 2021, quest'ultimo è stato rimosso nelle iterazioni successive, le quali hanno portato alla versione che attualmente è standard: \citetitle{ieeeStandard3302-2022}.
    \end{onehalfspace}
    \item \begin{onehalfspace}
        Potenziare l'\ac{AIM} Tape Audio Restoration abilitandolo a modificare, magari tramite l'uso di \acl{IA}, l'audio della copia di accesso in corrispondenza delle irregolarità video grazie alle informazioni sulla loro tipologia. Tali discontinuità possono aver deteriorato la resa acustica del nastro.
    \end{onehalfspace}
\end{itemize}

L'aspetto negativo di queste proposte è la diminuzione della fedeltà dell'audio in output, ma ciò è completamente accettabile se operato solo e soltanto sulla copia di accesso.
