%!TEX root = ../main.tex

\chapter{MPAI} \label{chp:mpai}

% TODO decidere se nominare Chiariglione
Dalle ceneri del rinomato \ac{MPEG}, nel luglio 2020 nasce \ac{MPAI}\footnote{\url{https://mpai.community}}.

\ac{MPAI} è un'organizzazione non profit che, ancora guidata da Leonardo Chiariglione, ha come obiettivo la promozione dell'uso efficiente dei dati\footnote{Per dati \ac{MPAI} intende, per esempio, dati mediatici, manifatturieri, automobilistici, sanitari e generici. \cite{mpaiMPAICommunity}.} tramite lo sviluppo di specifiche tecniche per la codifica di qualunque tipo di dato facendo uso dell'intelligenza artificiale e la semplificazione dell'utilizzo di tali codifiche imponendo ai detentori di proprietà intellettuale di stabilire delle licenze per framework (\acs{AIF}), invece di dover essere legati ai brevetti e creare delle \textit{patent pool} \cite{mpaiMPAICommunity}. Sostanzialmente l'organizzazione si pone come missione quella di porre ordine nel mondo delle codifiche utilizzanti l'IA e di farlo semplificando il metodo di accesso alle proprie tecnologie rispetto ad \ac{MPEG}.
\ac{MPAI} opera attraverso la collaborazione delle varie parti interessate, tra cui l'università di Padova tramite il suo spin-off \href{www.audioinnova.com}{Audio Innova}.

L'utilizzo dell'intelligenza artificiale in questi tempi sta crescendo in maniera esponenziale e si sta avvicinando all'utente comune tramite la miriade di piattaforme online che sono nate, un esempio è \href{https://chat.openai.com}{ChatGPT} che ha raggiunto quota 100 milioni di utenti attivi in soli 2 mesi, raggiungendo il primato di applicazione ad uso privato con la crescita più veloce della storia.\footnote{\url{https://www.reuters.com/technology/chatgpt-sets-record-fastest-growing-user-base-analyst-note-2023-02-01/}} Il problema "nascosto" dell'IA è il fatto che è una tecnologia difficilmente comprensibile dalle masse nonostante il suo, come visto, uso spropositato; ciò porta l'utente a non \textit{capire} che il chatbot di turno non replichi utilizzando un principio di causalità, ma piuttosto seguendo dei pattern linguistici che non sempre portano ad avere risposte corrette, nonostante l'autorevolezza col quale sembri scrivere.

Il comitato di \ac{MPAI} si impegna ad affrontare i quesiti etici derivanti dal suo operato col coinvolgimento di esperti esterni; tali quesiti sono molto rilevanti a causa del rapido e soltanto recente sviluppo dell'IA.

Il progetto comprende diverse aree d'effetto tra cui il dialogo uomo-macchina, l'esperienza audio, la compressione video, l'esperienza di gioco online, la creazione di esperienze collaborative nel metaverso, la codifica di dati sanitari, i veicoli a guida autonoma e molti altri ed è in continua espansione.\footnote{Tutti i vari progetti sono visualizzabili sul \href{https://mpai.community/standards/}{sito di \ac{MPAI} alla voce \textit{Standards}}.}


\section{MPAI-AIF, AIW e AIMs} \label{sec:aif-aiw-aim}
Ogni standard \ac{MPAI} è un \ac{AIF} \cite{mpaiMPAIAIFMPAICommunity}: un ambiente che comprende diversi \ac{AIW}, ognuno che descrive un certo caso d'uso. I blocchi costituenti un workflow sono detti \ac{AIM} ed ogni modulo è definito dalla sintassi e dalla semantica delle proprie interfacce di input e output, l'implementazione (hardware o software che sia) non è specificata; i vari moduli svolgono delle specifiche attività e sono interconnessi a formare un AIW come si può vedere dalla figura \ref{fig:mpai-aif-architecture}.

\acs{AIF}, il modello fondante gli altri standard dell'organizzazione è stato adottato dall'\ac{IEEE} col nome \textit{IEEE 3301-2022} \cite{ieeeStandard3301-2022}.

\begin{figure}[h]
    \centering
    \includegraphics[width=\textwidth]{mpai-aif-architecture.png}
    \caption{Architettura di \acs{AIF} \cite{leonardoBlogNewWayDevelop2020}}
    \label{fig:mpai-aif-architecture}
\end{figure}


\section{Struttura di uno standard MPAI} \label{sec:standard-mpai} % e come implementarlo
Lo sviluppo di uno standard MPAI segue le fasi mostrate in figura \ref{fig:mpai-standard-stages}.    % TODO scrivere di più?

\begin{figure}[h]
    \centering
    \includegraphics[width=\textwidth]{mpai-standard-stages.png}
    \caption{Le fasi dello sviluppo di uno standard \acs{MPAI} \cite{leonardoBlogNewWayDevelop2020}}
    \label{fig:mpai-standard-stages}
\end{figure}

Uno standard \ac{MPAI} è composto da un insieme di 4 documenti con i relativi software e dataset \cite{mpaiStructureMPAIStandards}:
\begin{description}
    \item[Specifiche tecniche (\textit{Technical Specification})] Contiene le norme che un'implementazione conforme deve necessariamente seguire; solitamente è un insieme di casi d'uso. Per ogni caso d'uso viene specificato l'\ac{AIW} che lo implementa con le funzioni che esegue, la sintassi e la semantica dei suoi dati in input ed output; la topologia degli \ac{AIM} costituenti l'\ac{AIW} e, per ogni \ac{AIM}, la loro funzione e la sintassi e la semantica dei loro input ed output. 
    % TODO scambiare di posto con aif, aiw, aim?
    \item[Software di riferimento (\textit{Reference Software})] Contiene il codice sorgente dell'implementazione delle specifiche tecniche dell'\ac{AIF} e dei suoi \ac{AIW} esponendo le interfacce dei propri \ac{AIM}. Inoltre il software deve essere fornito di un metodo per l'uso, la sua documentazione necessaria\footnote{Una \ac{KB}} ed eventualmente dei dati di esempio.
    \item[Test di conformità (\textit{Conformance Testing})] Un insieme di vincoli relativi all'output generato da un dato input a cui un'implementazione deve sottostare per essere definita conforme. Questo documento viene trattato in maniera più approfondita al capitolo \ref{chp:conformancetesting}.
    \item[Valutazione delle prestazioni (\textit{Performance Assessment})] Definisce gli attributi di Affidabilità (rispetto dello standard), Robustezza (capacità di gestione di nuovi dati), Equità (IA unbiased\footnote{Un'intelligenza artificiale può essere \textit{biased}, ovvero può "avere pregiudizi"; nel caso ottimo sono molto limitati (IA unbiased) perchè portano ad assunzioni errate.}) e Replicabilità (della valutazione) che vengono utilizzati per attribuire un voto all'implementazione (eventualmente dipendente da un certo dominio di applicazione).
\end{description}

\begin{adjustbox}{width=\textwidth}
    \begin{tikzpicture}[
        state/.style={rectangle, draw=black, minimum size=5mm},
    ]
        %Nodes
        \node[state]  (TS)                    {Specifiche tecniche};
        \node[state]  (RS)    [right=of TS]   {Software di riferimento};
        \node[state]  (CT)    [right=of RS]   {Test di conformità};
        \node[state]  (PA)    [right=of CT]   {Valutazione delle prestazioni};
        
        %Lines
        \draw[->] (TS.east) -- (RS.west);
        \draw[->] (RS.east) -- (CT.west);
        \draw[->] (CT.east) -- (PA.west);
    \end{tikzpicture}
\end{adjustbox} % TODO inserire in una figure?  % TODO troppo piccolo?


\section{MPAI-CAE} \label{sec:mpai-cae}
Tra i vari standard/framework, \ac{CAE} si occupa di utilizzare le informazioni sul tipo di esperienza audio vissuta dall'utente (intrattenimento, teleconferenza, restauro, ...) e 'informazione del contesto in cui si trova (a casa, in auto, in mobilità, in studio, ...) per agire sul contenuto dell'audio in input e fornire i risultati desiderati \cite{mpaiMPAICAE}.

Sono considerati 4 casi d'uso:
\begin{description}
    \item[\ac{EES}] Permette all'utente di indicare un'emozione per ottenere una versione di una traccia audio di parlato caricato con l'espressività specificata.  % TODO si dice "caricato"?
    \item[\ac{ARP}] Permette di creare copie di audio digitalizzato, valido per una conservazione a lungo termine e per una riproduzione corretta della registrazione.
    \item[\ac{SRS}] Permette di ripristinare un segmento danneggiato di traccia audio contenente il parlato di un singolo oratore sintetizzando la voce della parte corrotta.
    \item[\ac{EAE}] Permette di migliorare la qualità sonora in un'audioconferenza utilizzando i segnali registrati da array di microfoni rimuovendo rumori di fondo e artefatti acustici.
\end{description}

\ac{CAE} è stato adottato dall'\ac{IEEE} come standard \textit{IEEE 3302-2022} \cite{ieeeStandard3302-2022}.

Nel documento \citetitle{ieeeStandard3302-2022} si possono trovare tutte le informazioni sullo standard.


\subsection{MPAI-CAE-ARP} \label{ssec:mpai-cae-arp} % è l'implementazione del metodo del CSC, cos'è un codec, cosa fa, quali sono i suoi moduli
La procedura di digitalizzazione del \ac{CSC} introdotta nella sezione \ref{sec:csc-digitalizzazione} è stata proposta ad \ac{MPAI} ed è stata riconosciuta per la sua efficacia ed affidabilità, perciò è stata adottata come use case di \ac{CAE} col nome \acf{ARP}.

Il \ac{CSC}, con l'aiuto di vari collaboratori, ha prodotto anche un software di riferimento per \ac{ARP}, il quale non è altro che una codifica\footnote{Un codec (audio) è un software o un dispositivo che codifica o decodifica un segnale o uno stream di dati secondo una specifica convenzione.} audio lossless.

Nel 2023 Audio Innova è stata insignita del \textit{Cannes Neurons Award 2023 Palm d'Or} per il miglior progetto di utilizzo creativo di IA, \ac{ARP}, al World AI Cannes festival.\footnote{\url{https://mpai.community/2023/02/17/mpai-member-audio-innova-srl-received-the-cannes-neurons-award-2023-palm-dor/}}

Dati in input (vedi figura \ref{fig:arp-workflow}):
\begin{description}
    \item[Preservation Audio File] La copia digitalizzata dell'audio.
    \item[Preservation Audio-Visual File] Il file video prodotto dalla ripresa della testina di registrazione e del capstan (come in figura \ref{fig:tape-areas}).
\end{description}

Si ottengono in output (vedi figura \ref{fig:arp-workflow}):
\begin{description}
    \item[Access Copy Files] Il file audio restaurato, una lista delle modifiche effettuate, la lista delle irregolarità con relativa classificazione e le loro istantanee dal video.
    \item[Preservation Master Files] Il file audio di input, il file video con l'audio sostituito da quello registrato e sincronizzato, la lista delle irregolarità con relativa classificazione e le loro istantanee dal video.
\end{description}

\begin{figure}[H]
    \centering
    \includegraphics[width=\textwidth]{arp-workflow.png}
    \caption{\ac{AIW} di \acl{ARP}}
    \label{fig:arp-workflow}
\end{figure}

Facendo riferimento alla figura \ref{fig:arp-workflow} gli \ac{AIM} di \ac{ARP} sono:
\begin{description}
    \item[Audio Analyser] È l'\ac{AIM} che rileva le irregolarità nell'audio, estrae i segmenti di \qty{500}{\ms} in loro corrispondenza e li invia al classificatore.
    \item[Video Analyser] È l'\ac{AIM} che rileva le irregolarità nel video e cattura delle immagini in loro corrispondenza.
    \item[Tape Irregularity Classifier] È l'\ac{AIM} che classifica le irregolarità di audio e video a partire dalle irregolarità rilevate da audio analyser e video analyser.   % TODO nel codice è compreso in aa e va? -> chiedere
    \item[Tape Audio Restoration] È l'\ac{AIM} che corregge velocità, equalizzazione e registrazione a rovescio dell'audio.
    \item[Packager] È l'\ac{AIM} che produce Access Copy Files e Preservation Master Files a partire dai file ricevuti dall'output degli altri \ac{AIM}.
\end{description}

\textit{Informazioni tratte da \cite{mpaiMPAIDataCoding}; maggiori informazioni nelle specifiche tecniche \cite{ieeeStandard3302-2022} e nella video presentazione del software di riferimento \cite{mpaistandardsMPAIPresentsContextbased2023}}.   % TODO inserire come footnote al primo paragrafo?

Si osserva nella letteratura afferente al \ac{CSC} che la classificazione dei problemi dalla traccia audio ha un'accuratezza del $93,7\%$ (esempio in figura \ref{fig:confusion-matrix-audio-classification}) \cite[min. 35:10]{mpaistandardsMPAIPresentsContextbased2023}, mentre per la classificazione dalle immagini si raggiungono valori $\geq 98,9\%$ (esempio in figura \ref{fig:confusion-matrix-audio-classification}) \cite[fig. 3 e p. 70]{prettoComputingMethodologiesSupporting2018}.

\begin{figure}[h]
    \centering
    \begin{subfigure}{0.8\textwidth}
        \centering
        \includegraphics[width=\textwidth]{confusion-matrix-audio-classification.png}
        \caption{Matrice di confusione di classificatore audio \cite[min. 35:10]{mpaistandardsMPAIPresentsContextbased2023}}
        \label{fig:confusion-matrix-audio-classification}
    \end{subfigure}
    \par\bigskip
    \begin{subfigure}{0.9\textwidth}
        \centering
        \begin{subfigure}{0.45\textwidth}
            \centering
            \includegraphics[width=\textwidth]{confusion-matrix-video-classification-7dot5ips-tape.png}
            \caption{Matrice di confusione di classificatore video, esperimento a velocità \qty{7,5}{ips}}
            \label{fig:confusion-matrix-video-classification-7dot5ips-tape}
        \end{subfigure}
        \hfill
        \begin{subfigure}{0.45\textwidth}
            \centering
            \includegraphics[width=\textwidth]{confusion-matrix-video-classification-15ips-tape.png}
            \caption{Matrice di confusione di classificatore video, esperimento a velocità \qty{15}{ips}}
            \label{fig:confusion-matrix-video-classification-15ips-tape}
        \end{subfigure}
        \caption{Matrice di confusione di classificatore video \cite[fig. 3]{prettoComputingMethodologiesSupporting2018}}
        \label{fig:confusion-matrix-video-classification}
     \end{subfigure}
        \caption{Matrici di confusione\footnote{Una matrice di confusione, \textit{confusion matrix} è una rappresentazione visuale dell'accuratezza di un classificatore, ogni colonna rappresenta i valori predetti ed ogni riga i valori reali, in corrispondenza di ogni intersezione si trova il valore assoluto o la percentuale di volte in cui si è verificata tale intersezione.} da esperimenti in letteratura con classificatori per \ac{ARP}.}  % TODO non si vede la footnote
        \label{fig:confusion-matrix-audio-video}
\end{figure}

Le tipologie di irregolarità previste e riconosciute dallo standard sono riassunte nella tabella \ref{tab:arp-irrs}.

\begin{table}[h]
    \centering
    \begin{tabular}{|c|c|}
        \hline
        \textbf{Acronym} &      \textbf{Meaning}\\
        \hline
        B       &   Brands on tape\\
        DA      &   Damaged tape\\
        DI      &   Dirt\\
        EOT     &   Ends of tape\\
        ESV     &   Equalization standard variation\\
        M       &   Marks\\
        PPS     &   Play, pause, stop\\
        PSD     &   Power spectral density\\
        RMSE    &   Root Mean Square Error\\
        S       &   Shadows\\
        SB      &   Signal Backward\\
        SOT     &   Start of tape\\
        SP      &   Splice\\
        SSV     &   Speed standard variation\\
        WF      &   Wow and flutter\\
        \hline
    \end{tabular}
    \caption{Irregolarità rilevate da \ac{ARP}} \cite[tab. 21-22]{ieeeStandard3302-2022}
    \label{tab:arp-irrs}
\end{table}

% TODO scrivere di più? o basta quello scritto in csc-digitalizzazione; in caso aggiungere alla fine