\begin{abstract}[it]
L'uomo ha sempre sentito la necessità di tutelare le proprie opere; al giorno d'oggi la tecnica per preservare i manufatti dall'usura e dall'obsolescenza è la digitalizzazione.

Questo lavoro presenta lo stato dell'arte della digitalizzazione di audio analogico, nello specifico registrato su nastri magnetici a bobina aperta (open-reel tapes).
Uno dei massimi esponenti nell'ambito è il \acl{CSC} del \acf[0]{DEI} dell'Università degli Studi di Padova, il quale ha sviluppato un software atto a riconoscere le discontinuità presenti su un nastro magnetico e rimediare agli errori di velocità ed equalizzazione nella traccia audio registrata, automatizzando un lavoro ripetitivo precedentemente svolto da operatori umani.
Il software è ora parte di un framework appartenente al gruppo \acf{MPAI} ed è stato adottato dall'\acs{IEEE} come standard col nome \acs{CAE}-\acs{ARP}. 

Lo scopo di quest'opera è trasporre in codice i test di conformità richiesti da \acs{MPAI} utilizzando un approccio di \acl{TDD}.
Il testo descrive lo sviluppo dei test richiesti e come sono stati risolti alcuni problemi pre-esistenti nel codice.

\begin{flushright}
    \vspace*{\fill}
    \footnotesize\textit{Il codice sorgente citato ed utilizzato per quest'opera si trova nella repository GitHub:} \url{https://github.com/albertopasqualetto/Tesi-triennale}.
\end{flushright}
\end{abstract}